\documentclass{sig-alternate-05-2015}
\begin{document}

% ACM templates include ISBN and DOI...
\isbn{N.A.}
\doi{N.A.}

\title{CPE480 Assignment 2: Multi-Cycle Tangled }
\subtitle{Implementor's Notes}

\numberofauthors{3}
\author{
Gerard (Jed) Mijares\\
Ben Luckett\\
Nick Santini\\
       \affaddr{Department of Electrical and Computer Engineering}\\
       \affaddr{University of Kentucky, Lexington, KY USA}\\
    %   \email{\texttt{gami227@uky.edu}}
}

\maketitle
\begin{abstract}
This is an multi-cycled implementation of a processor for the Tangled instruction set.
\end{abstract}

\section{General Approach}

Our AIK specification of Tangled is largely based on the sample solution provided by Dr. Dietz, with a few changes. Namely, we added two additional possibilities for the first 4 bits of the instruction: \texttt{Start} and \texttt{Decode}. \texttt{0xd} was available, so we used that for \texttt{Start}. However, at this point, we were out of hex numbers for the first 4 bits. So, we combined the instructions that were originally categorized under both \texttt{0x7} and \texttt{0x8} to only begin with \texttt{0x7}, leaving \texttt{0x8} available for \texttt{Decode}.

Our processor is structured as a state machine. The first step is to use a case statement on the first 4 bits of the instruction. Sometimes, that is sufficient to distinguish which instruction we should use, but if not, we nest a second case statement on the second 4 bits of the instruction to further determine which instruction to use.

Since some instructions will require a second 16-bit word to process, we elect to simply store both of the next two words in the \texttt{Decode} state.

As recommended by Dr. Dietz, we make use of Verilog's \texttt{`define}s to name ranges of bits we frequently use and the OP code for each instruction.

We decided to run our Verilog locally, so a few commands differ from Dr. Dietz's web interface. A Makefile is included to facilitate running the project.

\vfill\pagebreak

\section{Testing}

Not very fleshed out yet, there's some code to test the branch instructions.

\section{Issues}

Qat instructions originally didn't increment the pc when we needed two words.

Branch instructions were a bit difficult.

\end{document}
